\documentclass[10pt,a4paper]{article}

\usepackage{float}
\usepackage{subfigure}
\usepackage{graphicx}
\usepackage{subfigure}
\usepackage{listings}
\usepackage{alltt}
\usepackage{multirow}
\usepackage{rotating}
\usepackage{courier}
\usepackage{epsfig,endnotes,url,alltt,listings}
\usepackage{color}
\usepackage{framed}


\usepackage{ETHkopf}
\usepackage[utf8]{inputenc}
\usepackage{booktabs}
\usepackage{tabularx}


\textwidth 160mm
\textheight 230mm
\oddsidemargin 0mm
\evensidemargin 0mm
\setlength{\parskip}{0.2cm}
\setlength{\parindent}{0mm}

\makeatletter
\long\def\unmarkedfootnote#1{{\long\def\@makefntext##1{##1}\footnotetext{#1}}}
\makeatother

\def\refname{Reading material}

\begin{document}

%\title{Design and Implementation of a Microsecond-scale Spark Scheduler}
\title{Microsecond-level Scheduling for Apache Spark}

\author{External Master Thesis Project for Lukas Drescher}
\date{March-September, 2018}

\maketitle


\ETHkopf{Department of Computer Science, ETH Zurich}{Laboratory for Software
Technology}

\section*{Introduction}
Apache Spark~\cite{2018-apache-spark} is one of the most popular distributed
data processing frameworks. It offers a unified high-level abstraction 
(Resilient Distributed Datasets~\cite{2011-nsdi-rdd}) in a single framework 
for multiple data processing paradigms such as MapReduce, graph computation,
stream analysis, machine learning, and SQL processing. Due to its high
performance, convenient/intuitive programming APIs (provide an illusion 
of a single machine), and support for multiple languages (Java, Scala, Python, 
R, etc.), Spark is a popular choice for data scientists and developers. 


Internally, Spark parallelizes a program or a job by breaking it into multiple 
independent \textit{tasks} that run in stages. These tasks are scheduled on
multiple executor machines to process a part of the data, and finally their
outputs are combined to generate the result. As the demand for low latency, 
interactive data processing increases, over the years, many efforts have been 
put to make the runtime of a task smaller. Apart from delivering interactivity 
and high performance, small tasks are also beneficial to maintain the fairness 
between long running and short running jobs, mitigate stragglers and 
skew issues, provide fast recoverability in a case of failure, and improve 
resource utilization~\cite{2013-hotos-tiny-tasks}. 


However, making tasks small and scheduling them present some
interesting challenges.
In order to support small tasks (at milli- and microseconds level), a 
scheduler not only needs to make hundreds of thousands of 
decisions per second, but it also has to manage
and execute thousands of tasks at a similar rate. A high-performance
scheduler design is an active research field where recent
efforts~\cite{2015-sosp-sparrow,2015-eurosys-borg,2013-eurosys-omega,
2016-osdi-firm,2017-atc-lowlat}
have focussed on (i) improving the throughput of the decision process with
centralized, decentralized, and hybrid scheduler designs; (ii) increasing
interactivity by low-latency scheduling; and 
(iii) improving the cluster utilization by high-quality task placement
decisions.
However, not enough focus has not been put to the \textit{task management} 
and the \textit{control plane logic} that involve launching, cordinating,
and executing tasks and then collecting the result~\cite{2017-atc-nimbus}.
Specifically, in a framework like Spark, there are multiple inefficiencies that stem from 
(i) the actual scheduler implementation;  
(ii) runtime/language related overheads~\cite{2017-hotos-runtime}; 
(iii) communication and synchronization overheads; and  
(iv) data access related overheads. 
These inefficiencies effectively limits its scheduling throughput to a few 
thousand tasks per second, while the ``core'' scheduler might be making 
hundreds of thousands task placement decisions per second.     


Recent advancements in the networking and storage hardware (100+ Gbps
RDMA, NVMe flash, PCM memory), efficient I/O in management runtimes (IBM 
DiSNI and DaRPC stacks), and the availability of high-performance distributed
storage solutions (e.g., Apache Crail) now enable us to build a high-performance
control path for the scheduler, which is the prime focus of this thesis.
Ousterhout et al. already analyzed the possibility of such a proposal five 
years ago (section 3.3)~\cite{2013-hotos-tiny-tasks}: 

\begin{quote}
\textit{``Today’s datacenter networks easily allow a RPC to complete within 
1ms. In fact, recent work showed that 10$\mu$s RPCs are possible in the short 
term [26]; thus, with careful engineering, we believe task launch overheads 
of 50$\mu$s are attainable. 50$\mu$s task launch overheads would enable even
smaller tasks that could read data from in-memory or from flash storage
in order to complete in milliseconds.''}
\end{quote}

To the best of our knowledge, we are not aware of any
general-purpose system that is scheduling and operating in the 
microsecond range while scheduling hundreds of thousands of tasks 
per second.
Achieving this performance requires support from the whole stack 
(devices, runtimes, the framework, operating system, etc.) which is 
only realized in the last couple of years. As we move towards 
the \textit{Microsecond-Scale Data Processing Era}, we have evidence 
that such system design and target performance are now
viable~\cite{2017-sosp-zygos,2017-killer}.

% Current state-of-the-art 
% implementations (e.g., Sparrow~\cite{2015-sosp-sparrow}) target to schedule 
% tasks at the 10-100s milliseconds granularity due to various practical 
% reasons. 
% Going below the millisecond to the microsecond-level limit (to the
% \textit{$\mu$Tasks}) presents some interesting considerations. These
% considirations are presented in the Figure~\ref{}. First, in order to 
% schedule tasks at the micro-second granularity, a schedular needs to 
% make millions of placement decisions per second. The focus of recent 
% works in this field have been to (i) improve the latency and throughput 
% of the decision process; and (ii) improve the quality of 
% decisions~\cite{omega,borg,sparrow}. 
% 
% 
% We expect modern state-of-the-art 
% schedular can support the desired performance level for $\mu$Tasks.
% Secondly, $\mu$Tasks need microsecond-level data access in a distributed 
% setting. Due to the recent advancements in the I/O hardware, modern 
% networking and storage can provide 100+ Gbps of bandwidth, 10-100$\mu$sec 
% of access latencies with millions of IOPS. This performance is now avaiable 
% to data processing system using projects like Apache Crail~\cite{}. 
% Apache Crail not only delivers very high performance for reading and
% writing input/output datasets, but for the temporary data (e.g., shuffle,
% broadcast, etc) as well. The last challenge, which is the prime focus 
% of this thesis, is to lauch, execute, and finish a $\mu$Task as fast 
% as possible. This is the last missing piece of the puzzle that enables 
% us to do microsecond-level \textit{end-to-end} scheduling in Spark.   


\section*{Goal and Scope}
The goal of this thesis is to design and implement a Microsecond-level
task scheduling framework for Apache Spark. The thesis work should focus on 
the \textit{mechanism} (how to launch a networked task) rather than the 
\textit{policy} (when and where to schedule). The thesis output should 
demonstrate what performance can be achieved without fundamentally altering 
the design of the Spark scheduler and execution framework (for example, 
moving from a centralized to distributed scheduling). The target scope of
deployment of the scheduler is in a Rack-scale computing, where 10-100s 
of closely connected machines execute multiple workloads in a 
high-performance environment. 
  
% deployment setup of such a scheduler is Rack-scale (10-100s of
% machines) workloads where closely connected servers execute     
% 
% 
%  The idea of such semi-research work is always to explore if  the idea is
% possible, under what conditions, and if not then explain why not. Is it 
% something fundamental that cannot be changed (like the speed of light) or 
% just a matter of time or a design constraint, etc. As we discussed today, just 
% doing the back-of-the-envelop calculation shows that scheduling in useconds 
% "should" be possible. May be it is only attenable for null-tasks or may be 
% it works also for general purpose tasks.
%  behaviour of
% the
% 
% the control plane where task l 
% 
% implementation should focus on demonstrating the feasibility of the approach
% in a real working system.
% To the best of our knowledge, there is no Spark schedular 
% that can operate at the microsecond level. 

\section*{Tasks}

More specifically, the student \textit{should} focus on following tasks to
improve the performance of the scheduling framework in Spark:

\begin{itemize}
  \item \textbf{Communication Overheads:} In order to support $<10\mu$s 
  task launch times, the communication layer in Spark (between the
  driver and executors) should be optimized for high-performance
  networks. Here, the focus of the work should be on porting the Spark's
  Netty/NIO based RPC communication framework to a highly-optimized RDMA
  based DaRPC implementation~\cite{2018-github-darpc,2014-socc-darpc}.

\item \textbf{Java/JVM Overheads:} Here, the focus should be on removing 
overheads that emerge from implementing the framework in a managed runtime
like JVM. These overheads might come from (de)serialization, data copies, 
object management, efficient Java/Scala codes, JITing issues, 
and garbage collection, etc. An example analysis can be referenced
here~\cite{2016-osdi-warmjvm}.
  
\item \textbf{Scheduler Implementation:} Lastly, the effort here should be
focused on the implementation-related overheads in the current Spark scheduler
code. There may be inefficiencies related to the concurrent 
data structures used, locking, notification, resource management, 
unnecessary code in the critical path, etc. General overheads from the 
inefficient distributed control path for long-running repetitive tasks 
are discussed here~\cite{2017-atc-nimbus}.
\end{itemize}

As we move further in the thesis work, we expect these directions to be 
revised and new directions to emerge with the input from implementation 
and performance profiling tools.  

\section*{General instructions}

\begin{itemize} 
  
  \item The project starts on March 13th, 2018.
  
  \item The project should be completed by September 11th, 2018.

  \item Prepare a thesis draft containing Introduction, Problem Statement and
  Related Work, and a work plan after a month.
  
  \item The progress of the thesis work should be discussed regularly with
  the supervisors at ETH and IBM Research.
  
  \item The deliverables of this work are (1) the source code, (2) a
  report, and (3) a one page summary.
  
  \item The report should be phrased as a scientific essay and in PDF format.
  It must be in English. 
  
  \item There should be an oral presentation at the end of the thesis
  work.
  
\end{itemize}


\section*{Grading scheme}

Grading will be based on the following criteria:

\begin{center}
\begin{tabular}{lr}
\toprule
Criterion & Weight ($\pm$ 0.5) \\
\midrule
Implementation (functionality, extensibility, documentation) & 3 \\
Report (content, illustration, writing) & 2 \\
Presentation (content, illustration, quality of talk) & 1 \\
Approach (organization, approaching problems, independence, involvement) & 1
\\
Contribution to the state of the art research & 1 \\
Completion of all tasks & 2 \\
\bottomrule
\end{tabular}
\end{center}

Each criterion will be graded in accordance with the
``Merkblatt zur Master-Arbeit in Informatik nach Studienreglement 09''.
%``Merkblatt zur Bachelor-Arbeit in Informatik nach Studienreglement 08''.


\textbf{ETH Professor:} Prof. Dr. Thomas R. Gross (trg@inf.ethz.ch)\\
\textbf{IBM Research Supervisor:} Animesh Trivedi (atr@zurich.ibm.com)\\

\section*{More Papers}
Papers about high-performance data processing in systems (single machine or 
distributed system) that might be of relevence to what we are trying to build. 
\begin{itemize}
  \item SOSP 2017: MittOS: Supporting Millisecond Tail Tolerance with Fast Rejecting SLO-Aware OS Interface
  \item SOSP 2017: Monotasks: Architecting for Performance Clarity in Data Analytics Frameworks
  \item SOSP 2017: LITE Kernel RDMA Support for Datacenter Applications
  \item SOSP 2017: ffwd: delegation is (much) faster than you think
  \item SOSP 2017: Drizzle: Fast and Adaptable Stream Processing at Scale
  \item SOSP 2017: Sub-millisecond Stateful Stream Querying over Fast-evolving Linked Data
  \item SOSP 2017: Low-Latency Analytics on Colossal Data Streams with SummaryStore  
\end{itemize}
%\pagebreak
\raggedright
%% BIBLIOGRAPHY
\bibliographystyle{unsrt}
\bibliography{main}
\end{document}
